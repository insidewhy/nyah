\documentclass{article}


\begin{document}

\title{A system for monitoring source code and updating binary code}
\author{James Pike}
\date{\today}

\maketitle

The approach taken by currently available compilers is to take a snapshot
of a subset of files within a given project and re-writing and re-linking all
modified compilation units entirely. No state information from previous
executions of the compiler is available. We will research an alternative 
model where the compiler is a daemon process which store state information 
which it can use to both optimise the build process and allow user queries to 
aid a development environment. Ultimately such a system could be used to
update only the minimal amount of compiled code based on changes to a source
tree and make these changes and the newly compiled code available almost 
instantaneously to many users working on a source tree.

\section*{Further Research Directions}


\bibliographystyle{plain}
\bibliography{thesisProposalBib}

\end{document}
