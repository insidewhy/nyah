\documentclass{article}

\begin{document}

\title{A Stateful Compiler running as a Daemon Process with Data-Mining Capabilities}
\author{James Pike}
\date{\today}

\maketitle

The approach taken by currently available compilers is to take a snapshot
of a subset of files within a given project and re-writing and re-linking all
modified compilation units entirely. No state information from previous
executions of the compiler is available. We will research an alternative 
model where the compiler is a daemon process which responds to changes in a
source tree and stores state information which it can use to both optimise the 
build process. Such a compiler would also be capable of responding to queries
about the source code to aid a development environment.

\section*{Further Research Directions}

Ultimately such a system could be used to compile and link only the minimal 
amount of code based on changes to a source tree and make these changes and 
the newly compiled code available almost instantaneously to many users 
working on a source tree simultaneously.

% \bibliographystyle{plain}
% \bibliography{thesisProposalBib}

\end{document}
